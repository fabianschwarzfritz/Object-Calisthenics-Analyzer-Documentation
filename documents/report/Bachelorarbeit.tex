% Präambel
\documentclass[11pt,a4paper,oneside, 
liststotoc, 					% Tabellen- und Abbildungsverzeichnis ins Inhaltsverzeichnis
bibtotoc,						% Literaturverzeichnis ins Inhaltsverzeichnis aufnehmen
titlepage, 						% Titlepage-Umgebung statt \maketitle
headsepline, 					% horizontale Linie unter Kolumnentitel
%abstracton,					% Überschrift beim Abstract einschalten, Abstract muss dazu in {abstract}-Umgebung stehen
%DIV11,							% auskommentieren, um den Seitenspiegel zu vergrößern
BCOR6mm,						% Bindekorrektur, die den Seitenspiegel um 6mm nach rechts verschiebt,
english
]{scrreprt}			
\usepackage{ucs} 				% Dokument in utf8-Codierung schreiben und speichern
\usepackage[utf8x]{inputenc} 	% ermöglicht die direkte Eingabe von Umlauten
\usepackage[ngerman]{babel} 	% deutsche Trennungsregeln und Übersetzung der festcodierten Überschriften
\usepackage[T1]{fontenc} 		% Ausgabe aller zeichen in einer T1-Codierung (wichtig für die Ausgabe von Umlauten!)
\usepackage{graphicx}  			% Einbinden von Grafiken erlauben
%\usepackage{amsmath}
%\usepackage{amsfonts}
%\usepackage{amssymb}
\usepackage{mathpazo} 			% Einstellung der verwendeten Schriftarten
\usepackage{textcomp} 			% zum Einsatz von Eurozeichen u. a. Symbolen
\usepackage{listings}			% Datstellung von Quellcode mit den Umgebungen {lstlisting}, \lstinline und \lstinputlisting
\usepackage{xcolor} 			% einfache Verwendung von Farben in nahezu allen Farbmodellen
\usepackage[intoc]{nomencl} 	% zur Erstellung des Abkürzungsberzeichnisses
\usepackage{fancyhdr}			% Zusatzpaket zur Gestaltung von Fuß und Kopfzeilen
\usepackage[htt]{hyphenat} 		%Line break of classnames
\usepackage{tikz}				%AST tree package
\usepackage{fullpage}
\usepackage{calc}
\usetikzlibrary{positioning,shadows,arrows,trees,shapes,fit}		%use the tree part of the package
% -----------------------------------------------------------------------------------------------------------------
% Zum Aktualisieren des Abkürzungsverzeichnisses bitte auf der Kommandozeile folgenden Befehl aufrufen :
%  makeindex Bachelorarbeit.nlo -s nomencl.ist -o Bachelorarbeit.nls
% -----------------------------------------------------------------------------------------------------------------

% Hier die persönlichen Daten eingeben:

\newcommand{\titel}{Validating the Object Calisthenics}
\newcommand{\untertitel}{Evaluation and Prototypical Implementation of Tool Support}
\newcommand{\arbeit}{Student Research Paper}
\newcommand{\studiengang}{Applied Computer Science}
\newcommand{\autor}{Fabian Schwarz-Fritz}
\newcommand{\matrikelnr}{212024979}
\newcommand{\kurs}{TINF11B2}
\newcommand{\firma}{SAP AG, Walldorf}
\newcommand{\abgabe}{\today}
\newcommand{\betreuerdhbw}{Daniel Lindner}
\newcommand{\betreuerfirma}{Soft­ware­schnei­de­rei GmbH, Karls­ru­he}

\newcommand{\jahr}{2014}			% für Angabe im Copyright-Vermerk der Titelseite

% Abkürzungen
\newcommand{\ua}{\mbox{u.\,a.\ }}
\newcommand{\zB}{\mbox{z.\,B.\ }}
\newcommand{\bs}{$\backslash$}

\renewcommand{\nomname}{Abbreviations}

% Code listing configuration: 
\newcommand{\classname}[1]{\texttt{#1}}
\lstset{numbers=left}
\lstset{backgroundcolor=\color{white}}
\lstset{frame=single}
\lstset{breaklines=true}
\lstset{morecomment=[l]{//}}

% -------------------------------------------------------------------------------------------
% Definition der Kopf- und Fußzeilen
\setlength{\headsep}{30pt}
\lhead{}								% Kopf links
\chead{}								% Kopf mitte
\rhead{\sffamily{\titel}}				% Kopf rechts
\lfoot{}								% Fuß links
\cfoot{\sffamily{\thepage}}				% Fuß mitte
\rfoot{\sffamily{\autor}}				% Fuß rechts
\renewcommand{\headrulewidth}{0.4pt}	% Liniendicke Kopf
\renewcommand{\footrulewidth}{0.4pt}	% Liniendicke Fuß


\makenomenclature							% Abkürzungsverzeichnis erstellen

% alle Abkürzungen, die in der Bachelorarbeit verwendet werden

\begin{acronym}
	\acro{DHBW}{Duale Hochschule Baden-Württemberg}
	\acro{Sem}{Semester}
	\acro{OSS}{Open Source Software}
	\acro{DIP}{Dependency Inversion Principle}
	\acro{HTML}{Hyper Text Markup Language}
\end{acronym}					% Datei mit Abkürzungen laden

% -------------------------------------------------------------------------------------------
%                     Beginn des Dokumenteninhalts
% -------------------------------------------------------------------------------------------
\begin{document}
\selectlanguage{english}					% Sprache englisch
\setcounter{secnumdepth}{3}					% Nummerierungstiefe fürs Inhaltsverzeichnis
\setcounter{tocdepth}{3}
\sffamily									% für die Titelei serifenlose Schrift verwenden

% ------------------------------ Titelei -----------------------------------------------------

\thispagestyle{plain}
\begin{titlepage}
\enlargethispage{4.0cm}
\sffamily 								% Serifenlose Grundschrift für die Titelseite einstellen
				
\begin{flushright}
\includegraphics[scale=2.0]{Bilder/logo_dhbw.jpg}\\[5ex]
\end{flushright}

\begin{center}

\huge{\textsc{\textbf{\titel}}}\\[1.5ex]
\Large{\textbf{\untertitel}}\\[5ex]
\LARGE{\textbf{\arbeit}}\\[2ex]
\normalsize{for the certification examinations for the\\[1ex] Bachelor of Science}\\[3ex]
\Large{Degree Course \studiengang}\\[1ex]
\normalsize{Baden-Wuerttemberg Cooperative State University Karlsruhe}\\[5ex]
by\\[1ex] \autor \\[18ex]


\end{center}

\begin{flushleft}

\begin{tabular}{ll}
Publication date:				& \quad \abgabe \\
Time required for processing:	& \quad 12 Weeks   \\ 
Matriculation number: 			& \quad \matrikelnr \\ 
Course: 						& \quad \kurs \\
Vocational training company:	& \quad \firma \\ 
Reviewer:  						& \quad \betreuerdhbw \\ 
Reviewer's company: 			& \quad \betreuerfirma \\ [5ex]

\end{tabular} 



\small
Copyrightvermerk:\\

Dieses Werk einschließlich seiner Teile ist \textbf{urheberrechtlich geschützt}. Jede Verwertung außerhalb der engen Grenzen des Urheberrechtgesetzes ist ohne Zustimmung des Autors unzulässig und strafbar. Das gilt insbesondere für Vervielfältigungen, Übersetzungen, Mikroverfilmungen sowie die Einspeicherung und Verarbeitung in elektronischen Systemen.
\end{flushleft}
\begin{flushright}
\copyright{} \jahr
\end{flushright}
\end{titlepage} 				% erzeugt die Titelseite
\pagenumbering{Roman}						% große, römische Seitenzahlen für Titelei
\addchap{Eidesstattliche Erklärung}
Unless otherwise indicated in the text or references, or acknowledged above, this thesis 
\begin{quote}
\textit{\titel} -\textit{ \untertitel }
\end{quote}
is entirely the product of my own scholarly work. This thesis has not been submitted either in whole or part, for a degree at this or any other university or institution. This is to certify that the printed version is equivalent to the submitted electronic one.\\[10ex]

Karlsruhe, den \today \\[4ex]


\rule[-0.2cm]{5cm}{0.5pt} \\

\textsc{\autor} \\[10ex]

%\hrule 
%\vspace*{1.0cm}
%\noindent \textbf{\Large{Sperrvermerk - TODO ??? INFORMIEREN. Gehoert das der SAP AG}}\\
%\normalsize
%Die Ergebnisse der Arbeit stehen ausschließlich dem auf dem Deckblatt aufgeführten Ausbildungsbetrieb zur Verfügung. 				% Einbinden der eidestattlichen Erklärung
\chapter*{Abstract} %*-Variante sorgt dafür, das Abstract nicht im Inhaltsverzeichnis auftaucht
Jeff Bay’s „Object Calisthenics” is an exercise to improve the quality of object oriented code. Good object oriented code is hard to learn when coming from procedural code. Many developers think object oriented – but do they really write good object oriented software?
\\

The rules of the Object Calisthenics are presented in The ThoughtWorks Anthology by Jeff Bay. The rules train developers to enhance their object oriented coding style. The calisthenics are composed of nine rules that the developer has to stick with. Behind every rule there is a purpose why the rule is important and why it leads to better object oriented code.
\\

Usually, a developer doesn’t use these rules in real world projects but applies them in short two day exercises in which he designs and implements minimalistic software with little requirements. This could be a Minesweeper or a TicTacToe game for example. These training challenges should lead the developer to write better code and be more aware of code quality in real world projects.
\\

But when completing the training challenge the developer has to observe his own code and check if his own coding style satisfies the nine rules of the Object Calisthenics. Tool support could shorten the time of the training and furthermore guarantee that the developer sticks to the given rules.
The academic evaluation of a tool validating the Object Calisthenics and the prototypical implementation of such a tool is the objective of this report.
In this student research paper, the development of tool support for the Object Calisthenics is evaluated. In the course of the paper the rules are described, their validation is evaluated and prototype that validates the rules is implemented.
\\

The first part compromises the explanation of patterns and principals behind the rules. What is the architectural problem, addressed by the rules? What are the patterns and principals behind the rules? How do these patterns and principles lead to the rules?
\\

The discussion of the challenges validating the compliance is the second and main part of this contribution. What challenges occurred when validating the source code structure? For which rules was it not possible to find an implementation, determining the validity of a rule and what detained it? 
\\

In the third part of the paper, the development of a prototypical validation tool is described. The implementation validates the rules with the algorithms that were postulated. The implementation of at one rule validation is exemplified in this third part.   				% Einbinden des Abstracts

\tableofcontents							% Erzeugen des Inhalsverzeichnisses
\printnomenclature[2.0cm]					% Erzeugen des Abkürzungsverzeichnisses
\listoffigures 								% Erzeugen des Abbildungsverzeichnisses 
\listoftables 								% Erzeugen des Tabellenverzeichnisses
\pagebreak

% --------------------------------------------------------------------------------------------
%                    Inhalt der 
%---------------------------------------------------------------------------------------------
\pagenumbering{arabic}						% arabische Seitenzahlen für den Hauptteil
\pagestyle{fancy}					
\rmfamily

%\chapter{Introduction}
\label{cha:Einleitung}

\section{Motivation}
\label{sec:Motivation}


\section{Goals of this report}
\label{sec:ZielDerArbeit}

\section{Structure of this report}
\label{sec:AufbauDerArbeit}




%\include{Inhalt/vorlagen}
\chapter{Introduction}
\label{Introduction}
The Object Calisthenics are nine programming rules helping to write good object oriented code.

The paper capter "The Object Calisthenics" describes the purpose, the outcome and the rules. It was released in 2008 in the paper "The thoughtworks anthology. Essays on Software Technology and Innovation"\cite[p. 70-79]{oc2008}. The whole paper consists of thirteen chapters discussing various topics and ideas on how to improve software development. On of the paper's chapter describes the rules of the Object Calisthenics and their purpose and outcome, shortly.

All essays in the paper are written by developers working at the company "Thoughtworks inc". ThoughtWorks is well known for creating, designing and supporting high quality software. The company is to be said to be one of the most future oriented company in terms of technology and software principals. They describe themselves as "[\dots]a software company and community of passionate inidividuals whose purpose is to revolutionize software design, creation and delivery, while advocating for positive social change[\dots]"\cite{twWeb}

\section{Exercizing Better Object Oriented Programming Skills: a Concept by Jeff Bay}
The Object Calisthenics are an exercise to improve the quality of Object Oriented code. Good Object Oriented Code is hard to learn when coming from procedural code. Many developers think in Object Oriented code – but do they really write good Object Oriented Software? That is the question that Jeff Bay poses in his essay. 
Usually the developer doesn't use these rules in real world project but applies them in short two hour exercises in which he designs and implements minimalistic software with little requirements. This could be a Minesweeper or a TicTacToe game for example. These training challenges should lead the developer to write better code and be more aware of code quality in real world projects.

RESEARCH,l page 22: spen 20 hours, 1000 loc, break habits, etc....

With these little training sessions, the Object Calisthenics help to create highly object oriented code  in small projects. When applying the rules, the developers automatically fulfill many important software patterns and principals leading to higher code quality than the code would have without the given rules. By training developers to focus the rules they automatically apply various helpful and important software principals and software patterns. 

Add: RESEARCH page 5: the excercise, strict!

However the idea is that the developers recoginzes the value of the resulting code and then applies parts of the rules or the principals behind them to larger, real world projects. 
Improving the quality of software's implementation by little training sessions - that is the basic idea. Jeff Bay, the author of the paper, included this idea also in the name of the rules. The word "Calisthenics" undistinctable describes the approach, the idea and the outcome of the exercise.

\section{Tool Support to Validate the Object Calisthenics}
The purpose of the Object Calisthenics was already described. However, in this paper the focus lies on the evaluation and prototypical implementation of tool support, validating the rules of the Object Calisthenics. 

The outcome of tool support validating the source code written during a Object Calisthenics session is the following. The tool support might shorten the time of the training and furthermore guarantees that the developer sticks to the given rules. 

- Now what is the outcome of tool support for the Object Calisthenics?

Therefore, the next chapter  \ref{Description} describes the rules Object Calisthenics to understand the software principals and quality metrics behind the Object Calisthenics. 
The ensuing chapter \ref{Evaluation} discusses tool support for every rule. 

\chapter{Object Calisthenics by Jeff Bay - Patterns and Principals}
\label{Description}
This chapter describes the patterns and principals behind the Object Calisthenics. 
- general introduction and introduciton o fpaper. Qualities, research stuff.

===RESEARCH===
poorly written code, procedural code

Desribe each very shortly: 

no reusabilaity

hard to maintain

overview

structure

no bundle o data and behaviour

no modularity

not understandable

maintainability is hard

OO saves us! --> But ... because

Comparision of procedural versus object oriented programming
procedural: step by step, seldom informaiton hiding, actions mainulate data. Actions are spread all ove rthe programming

oo: "bundle" with capabilities, hides structures, delegates tasks that are not done by the object itself to other objects, objects model real world behaviour, decoupled and separated in different modules. "what" leads to "how": encapsulation, abstraction, inheritance, polymorphism

advantages of oop: modules, reusable maintainable, simplicity (describe each shortly) 

Describe each quality, a bit more detailes, but also short and concise: 

cohesion

loose coupling

zero duplication

encapsulation

testability

readability

focus

\section{Advantages of Object Oriented Programming}

\section{The Rules and their Background}
This chapter describes the rules of the Object Calisthenics itself. 

Repeat form introduciton:
The exercise: strict coding standards etc... RESEARCH page 5

Every rule is described one after the other. For every rule there will be ??? infrmation??? . This is a short explanation, a good and a bad example. Furthermore every chapter describes the software patterns and software principals behind every rule. These are for example design patterns, software principals and best practices. The outcome of every rule is then summarized.

For every rule 1-9: 
 - Explain the rule. Use quotes of paper. (research)
 - Good example with explanation
 - Bad example with explanation
 - Describe idea and principal behind the pattern (research). Refer to other sources with large stuff to explain and refer to previous chapter if already explained...
 - Summarize the rule's purpose

\subsection{"Use One Level of Indentation per Method"}
asdf
\subsection{"Don’t Use the else Keyword"}
asdf
\subsection{"Wrap All Primitives andvlue."}
asdf
\subsection{"Use Only One Dot per Line"}
asdf
\subsection{"Don't Abbreviate"}
asdf
\subsection{"Keep All Entities Small"}
asdf
\subsection{"Don’t Use Any Classes with More Than Two Instance Variables"}
asdf
\subsection{"Use First-Class Collections"}
asdf
\subsection{"Don’t Use Any Getters/Setters/Properties"}
asdf
\section{Discussing the Rules}
\subsection{Similarities}
Categorize rules: Are there similarities from perspective of principle? ??? Together with next chapter?
\subsection{Precedence}
Own estimation: what's the most important rule? What do I think? What does the author think? Reason with descriptions and examples given
\subsection{Conclusion and Outcome}
Give an outcome of the rules. RESERACh page 21: 
behaviour and operation oriented
LoD and loose coupling
talk to friends
talk the protocol specified by the object's operation
Next page: conslucstion, no else ,naming, 
all in all: duplication of code and idea
==> "Simple and elegant Abstractions"



\chapter{Tool Support to Validate the Object Calisthenics - Evaluation}
\label{Evaluation}
\section{Tool Support and It's Advantage}
What is a tool? Why do tools help? What makes tools strong? Why do tools matter for developers?
Possible outcome of tool support for the OC's?
\section{Working Environment}
Describe ast generally. Say that Eclipse provides types representing the parts of code syntax. This is seen as given. 
Refer to other references.
Describe shortly how it is possible to do an AST validation with eclpise. Say that: "Eclipse" terms and standart terms are use. These are not further described in this paper. Give reference for questions about "parameter", "type", "class", "expression" or "statement".
Say that the validaiton in the next section is exactly implemented as described. One (???) example is shown in the Prototype chapter.
\section{Evaluation of Rule Validation}
Say that the priorization of the rules and the "ranking" is given in the end. This chapter is "rule specific", even if the next subsections refer to each other.

Foreach: 
 - Similarities found out in description may be similar in this validation?
 - Cathegorize the rules in groups form a validation perspecitve
 - Use examples given in description chapter to describe the typical structure of the rule. 
 - Explane "the positive case": What is the positive structure, satisfying the rule
 - What checks have to be done for a possible validation
 - Discussion of 'rule dependencies' within one rule (example: wrapper has to determine possible wrapper classes first...)
 - --> solution found/no solution found
 
 - Be self-critical: Now, were a solution is found (or not), describe the problems that occur with the described implementation

\subsection{"Use One Level of Indentation per Method"}
asdf
\subsection{"Don’t Use the else Keyword"}
asdf
\subsection{"Wrap All Primitives andvlue."}
asdf
\subsection{"Use Only One Dot per Line"}
asdf
\subsection{"Don't Abbreviate"}
asdf
\subsection{"Keep All Entities Small"}
asdf
\subsection{"Don’t Use Any Classes with More Than Two Instance Variables"}
asdf
\subsection{"Use First-Class Collections"}
asdf
\subsection{"Don’t Use Any Getters/Setters/Properties"}
asdf
\section{Result of the Evaluation}
Give a summary on how hard it was to implement the rules
\section{Future Work}

\chapter{Prototypical Implementation of Tool Support}
\section{Requirements}
Describe requirements for the prototype
\section{Architecture}
Describe overall architecture (???packages)
\section{Exemplary Rule Validation}
One example implementation of one rule validation
\section{Resulting Prototype}
Show screenshot and describe UI. What is possible, what is not possible. 
What are ideas that are still out there? 
How could the product improve?
\section{Outlook and Future Work}


%\textbf{}
%\begin{figure}[htb]
%\centering
%\includegraphics[width=0.8\textwidth]{FHWTLogo.jpg}
%\caption{Das Logo der FHWT}
%\label{fig:FHWTLogo}
%\end{figure}


%\include{Inhalt/zusammenfassung}



% ---------------------------- Literaturverzeichnis ----------------------------------------------

\begin{thebibliography}{999999}

\bibitem[FoBa03]{foobar2003} Foo, John; Bar, Belinda: \emph{Titel : Untertitel},\\ Verlagsort: Verlag, Jahr der Auflage. S. 10-20
\bibitem[Le01]{levy2001} Autor Name: \emph{Titel des Buches}, New York: Penguin Books, 2001.

LITERATUR: 

\bibitem[Bay, 2008]{bay2008} Bay, Jeff: 
\emph{Object Calisthenics}. \\ In: ThoughtWorks inc. (eds): 
\emph{The ThoughtWorks Anthology. Essays on Software Technology and Innovation}. \\ Raleigh, North California; Dallas, Texas: The Pragmatic Bookshelf, 2008, p. 70--80.

\bibitem[ThoughtWorks inc., 2008]{oc2008} ThoughtWorks inc. (eds): 
\emph{The ThoughtWorks Anthology. Essays on Software Technology and Innovation}. \\ Raleigh, North California; Dallas, Texas: The Pragmatic Bookshelf, 2008.

\bibitem[Gamma et al., 1994]{gof} Gamma, Eric; Helm, Richard; Johnson, Ralph; Vlissides,  John:
\emph{Design Patterns. Elements of Reusable Object-Oriented Software}. \\ Amsterdam: Addison-Wesley Longman, 1994.
  
\bibitem[Martin, 2008]{cc} Martin, Robert Cecil:
\emph{Clean Code. A Handbook of Agile Software Craftsmanship}. \\ n.p., Prentice Hall International, 2008.

\bibitem[Fowler et al., 1999]{ref} Fowler, Martin:
\emph{Refactoring. Improving the Design of Existing Code}. \\ Amsterdam: Addison-Wesley Longman, 1999. 

\bibitem[Evans, 2003]{ddd} Evans, Eric:
\emph{Domain-Driven Design. Tackling Complexity in the Heart of Software}. \\ Amsterdam: Addison-Wesley Longman, 2003. 

asdfklsajkflj INTERNET


\bibitem[Eclipse Documentation]{eclipseDocu} The Eclipse Foundation: 
\emph{Eclipse documentation - Eclipse Kepler}. \\ URL http://help.eclipse.org/kepler/index.jsp.

\bibitem[Wikipedia]{wiki} Wikipedia. The Free Encyclopedia. \\ URL http://www.wikipedia.org. 

\bibitem[About.com]{about} Haas, Juergen: \emph{Modular Programming}.\\ URL http://linux.about.com/cs/linux101/g/modularprogramm.htm.

\bibitem[ToughtWorks inc.]{twWeb} Thoughtworks inc. \\ URL thoughtworks.com/about-us  (17 November 2013).

\end{thebibliography}

% ------------------------------- Anhang ---------------------------------------------------------

\begin{appendix}
\clearpage
\pagenumbering{Roman}						% römische Seitenzahlen für Anhang
\end{appendix}


\end{document}
