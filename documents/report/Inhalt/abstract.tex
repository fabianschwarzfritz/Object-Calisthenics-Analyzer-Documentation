\chapter*{Abstract} %*-Variante sorgt dafür, das Abstract nicht im Inhaltsverzeichnis auftaucht
Jeff Bay’s „Object Calisthenics” is an exercise to improve the quality of object oriented code. Good object oriented code is hard to learn when coming from procedural code. Many developers think object oriented – but do they really write good object oriented software?
\\

The rules of the Object Calisthenics are presented in The ThoughtWorks Anthology by Jeff Bay. The rules train developers to enhance their object oriented coding style. The calisthenics are composed of nine rules that the developer has to stick with. Behind every rule there is a purpose why the rule is important and why it leads to better object oriented code.
\\

Usually, a developer doesn’t use these rules in real world projects but applies them in short two day exercises in which he designs and implements minimalistic software with little requirements. This could be a Minesweeper or a TicTacToe game for example. These training challenges should lead the developer to write better code and be more aware of code quality in real world projects.
\\

But when completing the training challenge the developer has to observe his own code and check if his own coding style satisfies the nine rules of the Object Calisthenics. Tool support could shorten the time of the training and furthermore guarantee that the developer sticks to the given rules.
The academic evaluation of a tool validating the Object Calisthenics and the prototypical implementation of such a tool is the objective of this report.
In this student research paper, the development of tool support for the Object Calisthenics is evaluated. In the course of the paper the rules are described, their validation is evaluated and prototype that validates the rules is implemented.
\\

The first part compromises the explanation of patterns and principals behind the rules. What is the architectural problem, addressed by the rules? What are the patterns and principals behind the rules? How do these patterns and principles lead to the rules?
\\

The discussion of the challenges validating the compliance is the second and main part of this contribution. What challenges occurred when validating the source code structure? For which rules was it not possible to find an implementation, determining the validity of a rule and what detained it? 
\\

In the third part of the paper, the development of a prototypical validation tool is described. The implementation validates the rules with the algorithms that were postulated. The implementation of at one rule validation is exemplified in this third part.