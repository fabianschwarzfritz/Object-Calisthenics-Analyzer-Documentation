\lstinputlisting[label=listing:rule1example1,caption=Example 1: class \textit{School}]{listings/rule1/example1/School.java}
%\label{listing:rule1example1}

\lstinputlisting[label=listing:rule1example2,caption=Example 2: class \textit{School}]{listings/rule1/example2/School.java}
%\label{listing:rule1example2}

\lstinputlisting[label=listing:rule1example3:school,caption=Example 3: class \textit{School}]{listings/rule1/example3/School.java}
%\label{listing:rule1example3:school}
\lstinputlisting[label=listing:rule1example3:schoolclass,caption=Example 3: class \textit{SchoolClass}]{listings/rule1/example3/SchoolClass.java}
%\label{listing:rule1example3:schoolclass}

\lstinputlisting[label=listing:rule2examplenull:main,caption=Example 1 of class \textit{Main}]{listings/rule2/examplenull/Main.java}
%\label{listing:rule2example1}
\lstinputlisting[label=listing:rule2examplenull:mainnull,caption=Example 2 of class \textit{MainNull}]{listings/rule2/examplenull/MainNull.java}
%\label{listing:rule2example2}

\begin{lstlisting}[label=listing:nullobject1,caption={Null check, implemented by using a conditional}]
Object value = a.method();
if(value == null) {
	...
}
\end{lstlisting}
%\label{listing:nullobject1}

\begin{lstlisting}[label=listing:nullobject4,caption={Null check, implemented by using a conditional}]
Object value = a.method();
value.method()
\end{lstlisting}
%\label{listing:nullobject4}

\lstinputlisting[label=listing:rule3example1,caption=Example 1 of class \textit{Game}]{listings/rule3/example1/Game.java}
\lstinputlisting[label=listing:rule3example1caption=Example 1 of class \textit{Dice}]{listings/rule3/example1/Dice.java}
%\label{listing:rule3example1}


\lstinputlisting[label=listing:rule3example2,caption=Example 2 of class \textit{Game}]{listings/rule3/example2/Game.java}
\lstinputlisting[label=listing:rule3example2,caption=Example 2 of class \textit{Dice}]{listings/rule3/example2/Dice.java}
\lstinputlisting[label=listing:rule3example2,caption=New class \textit{Action} of example 2]{listings/rule3/example2/Action.java}
\lstinputlisting[label=listing:rule3example3:Day,caption={A typical wrapper class, representing a \textit{Day}}]{listings/rule3/example3/Day.java}
%\label{listing:rule3example2}


\lstinputlisting[label=listing:rule7example1:car,caption=Example 1 of class \textit{Car}]{listings/rule7/example1/Car.java}

\lstinputlisting[label=listing:rule7example2:car,caption=Example 2 of class \textit{Car}]{listings/rule7/example2/Car.java}
\lstinputlisting[label=listing:rule7example2:interior,caption=Example 2 of class \textit{Interior}]{listings/rule7/example2/Interior.java}
\lstinputlisting[label=listing:rule7example2:exterior,caption=Example 2 of class \textit{Exterior}]{listings/rule7/example2/Exterior.java}
\lstinputlisting[label=listing:rule7example2:centerconsole,caption=Example 2 of class \textit{CenterConsole}]{listings/rule7/example2/CenterConsole.java}
\lstinputlisting[label=listing:rule7example2:cockpit,caption=Example 2 of class \textit{Cockpit}]{listings/rule7/example2/Cockpit.java}


\lstinputlisting[label=listing:rule8example1:door,caption=Example 1 of class \textit{Exterior}]{listings/rule8/example1/Exterior.java}
\lstinputlisting[label=listing:rule8example1:door,caption=Example 1 of class \textit{Door}]{listings/rule8/example1/Door.java}


\lstinputlisting[label=listing:rule8example1:door,caption=Example 2 of class \textit{Exterior}]{listings/rule8/example2/Exterior.java}
\lstinputlisting[label=listing:rule8example1:door,caption=Example 2 of class \textit{Doors}]{listings/rule8/example2/Doors.java}
\lstinputlisting[label=listing:rule8example1:door,caption=Example 2 of class \textit{Door}]{listings/rule8/example2/Door.java}


\lstinputlisting[label=listing:rule9example1:main,caption=Example 1 of class \textit{Main}]{listings/rule9/example1/Main.java}
\lstinputlisting[label=listing:rule9example1:speaker,caption=Example 1 of class \textit{Speaker}]{listings/rule9/example1/Speaker.java}
\lstinputlisting[label=listing:rule9example2:main,caption=Example 2 of class \textit{Main}]{listings/rule9/example2/Main.java}
\lstinputlisting[label=listing:rule9example2:speaker,caption=Example 2 of class \textit{Speaker}]{listings/rule9/example2/Speaker.java}
\lstinputlisting[label=listing:rule9example2:mock,caption=Example 2 of class \textit{SpeakerMock}]{listings/rule9/example2/SpeakerMock.java}
\lstinputlisting[label=listing:rule9example3,caption=Example 3 of class \textit{Speaker}]{listings/rule9/example3/Speaker.java}

\lstinputlisting[label=listing:elsevisitor,caption=\textit{ASTVisitor} to validate rule two Don't Use the else Keyword]{listings/ElseVisitor.java}

