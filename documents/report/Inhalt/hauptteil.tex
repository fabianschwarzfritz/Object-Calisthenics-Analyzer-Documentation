\chapter{The Object Calisthenics}
For every rule 1-9: 
 - Explain the rule. Use quotes of paper. (research)
 - Good example with explanation
 - Bad example with explanation
 - Describe idea and principal behind the pattern (research)
 - Summarize the rule's purpose
\section{Purpose of the Object Calisthenics}
asdf
\section{The Rules}
asdf
\subsection{Use only one level of indentation per method}
asdf
\subsection{Don’t use the else keyword}
asdf
\subsection{Wrap all primitives and strings}
asdf
\subsection{Use only one dot per line}
asdf
\subsection{Don't abbreviate}
asdf
\subsection{Keep all entities small}
asdf
\subsection{Don't use any classes with more than two instance variables}
asdf
\subsection{Use first-class collections}
asdf
\subsection{Don’t use any getters/setters/properties}
asdf
\section{Similarities of the rules}
Categorize rules: Are there similarities from perspective of principle? ??? Together with next chapter?
\section{Precedence of the rules}
Own estimation: what's the most important rule? What do I think? What does the author think? Reason with descriptions and examples given
\section{Conclusion and outcome of the rules}
Give an outcome of the rules.

\chapter{Evaluation of tool support}
\section{??? Generatal tool introduction}
What is a tool? Why do tools help? What makes tools strong? Why do tools matter for developers?
\section{Introduction}
Possible outcome of tool support for the OC's?
\section{Abstract syntax tree}
Describe ast generally. Say that Eclipse provides types representing the parts of code syntax. This is seen as given. 
Refer to other references.
Describe shortly how it is possible to do an AST validation with eclpise. Say that: "Eclipse" terms and standart terms are use. These are not further described in this paper. Give reference for questions about "parameter", "type", "class", "expression" or "statement".
Say that the validaiton in the next section is exactly implemented as described. One (???) example is shown in the Prototype chapter.
\section{Evaluation of rule validation? TODO better title}
Say that the priorization of the rules and the "ranking" is given in the end. This chapter is "rule specific", even if the next subsections refer to each other.

Foreach: 
 - Similarities found out in description may be similar in this validation?
 - Cathegorize the rules in groups form a validation perspecitve
 - Use examples given in description chapter to describe the typical structure of the rule. 
 - Explane "the positive case": What is the positive structure, satisfying the rule
 - What checks have to be done for a possible validation
 - --> solution found/no solution found
 
 - Be self-critical: Now, were a solution is found (or not), describe the problems that occur with the described implementation
\subsection{Use only one level of indentation per method}
\subsection{Don’t use the else keyword}
\subsection{Wrap all primitives and strings}
\subsection{Use only one dot per line}
\subsection{Don't abbreviate}
\subsection{Keep all entities small}
\subsection{Don't use any classes with more than two instance variables}
\subsection{Use first-class collections}
\subsection{Don’t use any getters/setters/properties}
\section{Prioritize}
Give a summary on how hard it was to implement the rules

\chapter{Prototypical implementation of tool support}
\section{Prototype requirements}
Describe requirements for the prototype
\section{Architecture}
Describe overall architecture (???packages)
\section{An example of rule validation}
One example implementation of one rule validation
\section{The resulting prototype}
Show screenshot and describe UI. What is possible, what is not possible. 
What are ideas that are still out there? 
How could the product improve?


%\textbf{}
%\begin{figure}[htb]
%\centering
%\includegraphics[width=0.8\textwidth]{FHWTLogo.jpg}
%\caption{Das Logo der FHWT}
%\label{fig:FHWTLogo}
%\end{figure}

